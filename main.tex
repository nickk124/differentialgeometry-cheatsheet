\documentclass[10pt,landscape]{article}
\usepackage{amssymb,amsmath,amsthm,amsfonts,commath,bm}
\usepackage{multicol,multirow}
\usepackage[landscape,margin=0.5in]{geometry}

\title{A Differential Geometry Basics Cheat Sheet}
\author{Nick Konz}

\setlength{\columnseprule}{0.4pt}


% styles
\theoremstyle{definition}
\newtheorem{definition}{Definition}[section]

\theoremstyle{remark}
\newtheorem*{remark}{Remark}


% commands
\newcommand{\R}{\mathbb{R}}
\newcommand{\Rn}{\mathbb{R}^n}
\newcommand{\Ci}{C^\infty}

\begin{document}

\begin{center}
     \Large{\textbf{A Differential Geometry Basics Cheat Sheet}} \\
     \small{By Nick Konz}\\
     \textit{This is designed to be a quick, yet rigorous introduction/reference to the basic principles found within differential geometry, covering only the bare minimum of topics needed.}
     \small{\textit{Largely adapted from Spivak's} {A Comprehensive Introduction to Differential Geometry, 3rd Edition} and \textit{Mendelson's} Introduction to Topology, 3rd Edition}
\end{center}
\begin{multicols*}{3}

\section{Prerequisite Knowledge}
\textit{We only assume basic knowledge of continuity and differentiability in the context of functions and topology. Here, $\forall$ means ``for all'' or ``for each'', and $\exists$ means ``there exists''.}
\subsection{Some Basic Topology}

\theoremstyle{definition}
\begin{definition}{\textit{Neighborhood.}}
Let $a\in\Rn, \delta > 0$. The $\delta-$\textbf{neighborhood} of $a$ is the set
\begin{equation}
    U(a,\delta)=\left\{x\in\Rn:\norm{x-a}<\delta\right\}.
\end{equation}
\end{definition}

\theoremstyle{definition}
\begin{definition}{\textit{Metric Space.}}
Let $X$ be some non-empty set and $d$ be the mapping/function $d:X\times X\rightarrow \R$. The pair $(X,d)$ is a \textbf{metric space} if, $\forall x,y,z\in X$,
\begin{enumerate}
    \item $d(x,y) \geq 0$
    \item $d(x,y) = 0$ if and only if (iff) $x = y$
    \item $d(x,y)=d(y,x)$
    \item $d(x,z)\leq d(x,y) + d(y,z)$ \textit{(triangle inequality)}.
\end{enumerate}
\end{definition}

\theoremstyle{definition}
\begin{definition}{\textit{Topological Space.}}
Let $X$ be a non-empty set and $\mathcal{J}$ be a collection of subsets of $X$. $(X, \mathcal{J})$ is a \textbf{topological space} if
\begin{enumerate}
    \item $X \in \mathcal{J}$
    \item $\varnothing \in \mathcal{J}$, where $\varnothing$ is the empty set
    \item If $O_1, O_2, \ldots , O_n \in \mathcal{J}$, then $\bigcap_{i=1}^nO_i\in\mathcal{J}$
    \item If $\forall \alpha \in I$, $O_\alpha\in\mathcal{J}$, then $\bigcup_{\alpha\in I}O_\alpha\in\mathcal{J}$ \\($I$ is some \textit{indexing set}).
\end{enumerate}
We label $X$ as the \textbf{underlying set}, $\mathcal{J}$ as the \textbf{topology} on $X$, and members of $\mathcal{J}$ as \textbf{open sets}.

\end{definition}

\begin{remark}
Topological and metric spaces $(X,\mathcal{J})$ and $(X, d)$, respectively, are sometimes notated simply as $X$.
\end{remark}

\theoremstyle{definition}
\begin{definition}{\textit{Homeomorphism.}}
Let $(X,\mathcal{J})$ and $(Y, \mathcal{K})$ be topological spaces. $(X,\mathcal{J})$ and $(Y, \mathcal{K})$ are \textbf{homeomorphic} if $\exists$ inverse functions, or \textbf{homeomorphisms}, $f:X\rightarrow Y$ and $g:Y\rightarrow X$ such that $f, g$ are continuous.
\end{definition}

\section{Manifolds}

\theoremstyle{definition}
\begin{definition}{\textit{Manifold.}}
A \textbf{manifold} is a metric space $M$ such that if $x\in M$, $\exists$ some neighborhood $U$ of $x$ and some $n\in\left\{0,1,2\ldots\right\}$ such that $U$ is homeomorphic to $\Rn$. If $\exists$ such an $n$ that is the same $\forall x\in M$, we say that $M$ is \textbf{$\bm{n}$-dimensional}, which can be notated as $M^n$.
\end{definition}

\begin{remark}
Think of a manifold as being a surface that is locally Euclidean.
\end{remark}

\theoremstyle{definition}
\begin{definition}{\textit{$C^\infty$-related Homeomorphisms.}}
Let $M$ be some manifold, and let $U$, $V$ be open subsets of $M$. Two homeomorphisms $x:U\rightarrow x(U)\subset\Rn$ and $y:V\rightarrow y(V) \subset \Rn$ (for some $n$) are $\bm{\Ci}$\textbf{-related} if the maps
\begin{align}
    y \circ x^{-1} &: x(U\cap V) \rightarrow y(U\cap V)\\
    x \circ y^{-1} &: y(U\cap V) \rightarrow x(U\cap V)
\end{align}
are infinitely differentiable, or $\bm{\Ci}$.
\end{definition}

\theoremstyle{definition}
\begin{definition}{\textit{Atlas.}}
A family of \textit{mutually} $\Ci$-related homeomorphisms whose domains cover $M$ (i.e. their union equals $M$) is an \textbf{atlas} of $M$.
\end{definition}

\theoremstyle{definition}
\begin{definition}{\textit{Chart/Coordinate System.}}
A \textbf{chart} or \textbf{coordinate system} for some manifold $M$ is a homeomorphism $x$ from some open $U\in M$ to an open subset of $\Rn$, denoted $(x, U)$. A chart of $M$ is a member of some atlas of $M$.
\end{definition}

\begin{remark}
Charts/coordinate systems $(x, U)$ create a way of assigning coordinates to points on $U$, and are sometimes notated simply with $x$.
\end{remark}

\theoremstyle{definition}
\begin{definition}{\textit{Differentiable Manifold.}}
A \textbf{differentiable}, \textbf{smooth} or $\bm{C^\infty}$ manifold is a pair $(M, \mathcal{A})$, where $M$ is some manifold, and $\mathcal{A}$ is some \textit{maximal atlas} for $M$, i.e. the union of all possible atlases of $M$. 
\end{definition}

% add diffeomorphisms? only if needed later; this should be simple

\end{multicols*}
\end{document}