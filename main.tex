\documentclass[10pt,landscape]{article}
\usepackage{amssymb,amsmath,amsthm,amsfonts,commath}
\usepackage{multicol,multirow}
\usepackage[landscape,margin=0.5in]{geometry}

\setlength{\columnseprule}{0.4pt}


% styles
\theoremstyle{definition}
\newtheorem{definition}{Definition}[section]


% commands
\newcommand{\R}{\mathbb{R}}
\newcommand{\Rn}{\mathbb{R}^n}

\begin{document}

\begin{center}
     \Large{\textbf{A Differential Geometry Basics Cheat Sheet}} \\
     \small{By Nick Konz}\\
     \textit{This is designed to be a quick, yet rigorous introduction/reference to the basic principles found within differential geometry, covering only the bare minimum of topics needed.}
     \small{\textit{Largely adapted from Spivak's} {A Comprehensive Introduction to Differential Geometry, 3rd Edition} and \textit{Mendelson's} Introduction to Topology, 3rd Edition}
\end{center}
\begin{multicols*}{3}

\section{Prerequisite Knowledge}
\textit{We only assume basic knowledge of continuity and differentiability in the context of functions.}
\subsection{Some Basic Topology}

\theoremstyle{definition}
\begin{definition}{\textit{Neighborhood.}}
Let $a\in\Rn, \delta > 0$. The $\delta-$\textbf{neighborhood} of $a$ is the set
\begin{equation}
    U(a,\delta)=\left\{x\in\Rn:\norm{x-a}<\delta\right\}.
\end{equation}
\end{definition}

\theoremstyle{definition}
\begin{definition}{\textit{Metric Space.}}
Let $X$ be some non-empty set and $d$ be the mapping/function $d:X\times X\rightarrow \R$. The pair $(X,d)$ is a \textbf{metric space} if, $\forall x,y,z\in X$,
\begin{enumerate}
    \item $d(x,y) \geq 0$
    \item $d(x,y) = 0$ if and only if (iff) $x = y$
    \item $d(x,y)=d(y,x)$
    \item $d(x,z)\leq d(x,y) + d(y,z)$ \textit{(triangle inequality)}.
\end{enumerate}
\end{definition}

\theoremstyle{definition}
\begin{definition}{\textit{Topological Space.}}
Let $X$ be a non-empty set and $\mathcal{J}$ be a collection of subsets of $X$. $(X, \mathcal{J})$ is a \textbf{topological space} if
\begin{enumerate}
    \item $X \in \mathcal{J}$
    \item $\varnothing \in \mathcal{J}$, where $\varnothing$ is the empty set
    \item If $O_1, O_2, \ldots , O_n \in \mathcal{J}$, then $\bigcap_{i=1}^nO_i\in\mathcal{J}$
    \item If $\forall \alpha \in I$, $O_\alpha\in\mathcal{J}$, then $\bigcup_{\alpha\in I}O_\alpha\in\mathcal{J}$ \\($I$ is some \textit{indexing set}).
\end{enumerate}
We label $X$ as the \textbf{underlying set}, $\mathcal{J}$ as the \textbf{topology} on $X$, and members of $\mathcal{J}$ as \textbf{open sets}.

\end{definition}

\theoremstyle{definition}
\begin{definition}{\textit{Homeomorphism.}}
Let $(X,\mathcal{J})$ and $(Y, \mathcal{K})$ be topological spaces. $(X,\mathcal{J})$ and $(Y, \mathcal{K})$ are \textbf{homeomorphic} if $\exists$ inverse functions, or \textbf{homeomorphisms}, $f:X\rightarrow Y$ and $g:Y\rightarrow X$ such that $f, g$ are continuous.
\end{definition}

\theoremstyle{definition}
\begin{definition}{\textit{Manifold.}}
\end{definition}

\end{multicols*}
\end{document}