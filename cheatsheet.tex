\documentclass[10pt,landscape]{article}
\usepackage{amssymb,amsmath,amsthm,amsfonts,commath,bm}
\usepackage{multicol,multirow}
\usepackage[landscape,margin=0.5in]{geometry}

\title{A Differential Geometry Basics Cheat Sheet}
\author{Nick Konz}

\setlength{\columnseprule}{0.4pt}


% styles
\theoremstyle{definition}
\newtheorem{definition}{Definition}[section]

\theoremstyle{remark}
\newtheorem*{remark}{Remark}


% commands
\newcommand{\R}{\mathbb{R}}
\newcommand{\Rn}{\mathbb{R}^n}
\newcommand{\Ci}{C^\infty}

\begin{document}

\begin{center}
     \Large{\textbf{A Differential Geometry Basics Cheat Sheet}} \\
     \small{By Nick Konz}\\
     \textit{This is designed to be a quick, yet rigorous introduction/reference to the basic principles found within differential geometry, covering only the bare minimum of topics needed.}
     \small{\textit{Largely adapted from Spivak's} {A Comprehensive Introduction to Differential Geometry, 3rd Edition}, \textit{Mendelson's} Introduction to Topology, 3rd Edition and \textit{Michor's} Topics in Differential Geometry.}
\end{center}
\begin{multicols*}{3}

\section{Prerequisite Knowledge}
\textit{We only assume basic knowledge of continuity and differentiability in the context of functions and topology. Here, $\forall$ means ``for all'' or ``for each'', and $\exists$ means ``there exists''.}
\subsection{Some Basic Topology}

\theoremstyle{definition}
\begin{definition}{\textit{Neighborhood.}}
Let $a\in\Rn, \delta > 0$. The $\delta-$\textbf{neighborhood} of $a$ is the set
\begin{equation}
    U(a,\delta)=\left\{x\in\Rn:\norm{x-a}<\delta\right\}.
\end{equation}
\end{definition}

\theoremstyle{definition}
\begin{definition}{\textit{Metric Space.}}
Let $X$ be some non-empty set and $d$ be the mapping/function $d:X\times X\rightarrow \R$. The pair $(X,d)$ is a \textbf{metric space} if, $\forall x,y,z\in X$,
\begin{enumerate}
    \item $d(x,y) \geq 0$
    \item $d(x,y) = 0$ if and only if (iff) $x = y$
    \item $d(x,y)=d(y,x)$
    \item $d(x,z)\leq d(x,y) + d(y,z)$ \textit{(triangle inequality)}.
\end{enumerate}
\end{definition}

\theoremstyle{definition}
\begin{definition}{\textit{Topological Space.}}
Let $X$ be a non-empty set and $\mathcal{J}$ be a collection of subsets of $X$. $(X, \mathcal{J})$ is a \textbf{topological space} if
\begin{enumerate}
    \item $X \in \mathcal{J}$
    \item $\varnothing \in \mathcal{J}$, where $\varnothing$ is the empty set
    \item If $O_1, O_2, \ldots , O_n \in \mathcal{J}$, then $\bigcap_{i=1}^nO_i\in\mathcal{J}$
    \item If $\forall \alpha \in I$, $O_\alpha\in\mathcal{J}$, then $\bigcup_{\alpha\in I}O_\alpha\in\mathcal{J}$ \\($I$ is some \textit{indexing set}).
\end{enumerate}
We label $X$ as the \textbf{underlying set}, $\mathcal{J}$ as the \textbf{topology} on $X$, and members of $\mathcal{J}$ as \textbf{open sets}.

\end{definition}

\begin{remark}
Topological and metric spaces $(X,\mathcal{J})$ and $(X, d)$, respectively, are sometimes notated simply as $X$.
\end{remark}

\theoremstyle{definition}
\begin{definition}{\textit{Homeomorphism.}}
Let $(X,\mathcal{J})$ and $(Y, \mathcal{K})$ be topological spaces. $(X,\mathcal{J})$ and $(Y, \mathcal{K})$ are \textbf{homeomorphic} if $\exists$ inverse functions, or \textbf{homeomorphisms}, $f:X\rightarrow Y$ and $g:Y\rightarrow X$ such that $f, g$ are continuous.
\end{definition}

\subsection{Some Linear Algebra}
\begin{definition}{\textit{Vector Space.}}
A \textbf{vector space} $V$ is a set, or space, that is closed under addition and scalar multiplication, i.e. the result of performing these operations on some element(s) of $V$ is itself an element of $V$. We define elements of a vector space as \textbf{vectors}.
\end{definition}

\begin{definition}{\textit{Basis.}}
A \textbf{basis} for a vector space $V$ is a set of linearly independent vectors (i.e. none of the basis vectors can be written as linear combinations of the others) that span $V$ (i.e. any element of $V$ can be written as a linear combination of the basis vectors). In other words, the basis vectors define a ``coordinate system'' for $V$.
\end{definition}

\begin{remark}
You can convert vectors written in one basis to another via \textit{change-of-basis} matrices (or functions in the case of infinite-dimensional vectors).
\end{remark}

\subsection{Some Discrete Math}
\begin{definition}{\textit{Equivalence Relation.}}
A \textbf{relation} on some set $X$ is a subset $R$ of $X\times X$. We write $x R y$ to mean $(x, y)\in R$, i.e. \textit{x is related to y}.
The relation is an \textbf{equivalence relation} if it is
\begin{enumerate}
    \item \textbf{reflexive}: $aRa \quad \forall a\in X$
    \item \textbf{symmetric}: $aRb \Rightarrow bRa \quad \forall a,b\in X$
    \item \textbf{transitive}: $aRb,\, bRc \Rightarrow aRc \quad \forall a,b,c\in X$.
\end{enumerate}
\end{definition}

\begin{definition}{\textit{Equivalence Class.}}
Give an equivalence relation $R$ on some set $X$, the \textbf{equivalence class} of some $y\in X$ is the set $\left\{x\in X: xRy\right\}$.
\end{definition}

\section{Manifolds}

\theoremstyle{definition}
\begin{definition}{\textit{Manifold.}}
A \textbf{manifold} is a metric space $M$ such that if $x\in M$, $\exists$ some neighborhood $U$ of $x$ and some $n\in\left\{0,1,2\ldots\right\}$ such that $U$ is homeomorphic to $\Rn$. If $\exists$ such an $n$ that is the same $\forall x\in M$, we say that $M$ is \textbf{$\bm{n}$-dimensional}, which can be notated as $M^n$.
\end{definition}

\begin{remark}
Think of a manifold as being a surface that is locally Euclidean.
\end{remark}

\theoremstyle{definition}
\begin{definition}{\textit{$C^\infty$-related Homeomorphisms.}}
Let $M$ be some manifold, and let $U$, $V$ be open subsets of $M$. Two homeomorphisms $x:U\rightarrow x(U)\subset\Rn$ and $y:V\rightarrow y(V) \subset \Rn$ (for some $n$) are $\bm{\Ci}$\textbf{-related} if the maps
\begin{align}
    y \circ x^{-1} &: x(U\cap V) \rightarrow y(U\cap V)\\
    x \circ y^{-1} &: y(U\cap V) \rightarrow x(U\cap V)
\end{align}
are infinitely differentiable, or $\bm{\Ci}$.
\end{definition}

\theoremstyle{definition}
\begin{definition}{\textit{Atlas.}}
A family of \textit{mutually} $\Ci$-related homeomorphisms whose domains cover $M$ (i.e. their union equals $M$) is an \textbf{atlas} of $M$.
\end{definition}

\theoremstyle{definition}
\begin{definition}{\textit{Chart/Coordinate System.}}
A \textbf{chart} or \textbf{coordinate system} for some manifold $M$ is a homeomorphism $x$ from some open $U\in M$ to an open subset of $\Rn$, denoted $(x, U)$. A chart of $M$ is a member of some atlas of $M$.
\end{definition}

\begin{remark}
Charts/coordinate systems $(x, U)$ create a way of assigning coordinates to points on $U$, and are sometimes notated simply with $x$.
\end{remark}

\theoremstyle{definition}
\begin{definition}{\textit{Differentiable Manifold.}}
A \textbf{differentiable}, \textbf{smooth} or $\bm{C^\infty}$ manifold is a pair $(M, \mathcal{A})$, where $M$ is some manifold, and $\mathcal{A}$ is some \textit{maximal atlas} for $M$, i.e. the union of all possible atlases of $M$. 
\end{definition}

% add diffeomorphisms? only if needed later; this should be simple

\begin{remark}
To summarize, a manifold is essentially a space that can be covered with coordinate charts, which are invertible, continuous mappings to some subset of $\Rn$. If the mapping between any pair of overlapping charts is differentiable, then the manifold itself is differentiable.
\end{remark}

\section{The Tangent Bundle}

\theoremstyle{definition}
\begin{definition}{\textit{Tangent Space of $\Rn$.}}
Consider some point $v\in\Rn$, drawn as an arrow with a ``reference point'' of some $p\in\Rn$; this arrow from $p$ to $p+v$ is denoted $(p,v)$. The set of all such $(p,v)$ is the \textbf{tangent space} $T\Rn = \Rn\times\Rn$ of $\Rn$. Elements of $T\Rn$ are \textbf{tangent vectors} of $\Rn$.
\end{definition}

\begin{definition}{\textit{Projection Map on $\Rn$.}}
The \textbf{projection map} $\pi:\Rn\times\Rn\rightarrow\Rn$ recovers the first member of any pair $(p, v)\in T\Rn$, and is defined as
\begin{equation}
    \pi(p, v) = p.
\end{equation}
\end{definition}

\begin{remark}
The projection map can be thought of as mapping any tangent vector to ``where it's at''/``where it comes from''.
\end{remark}

\begin{definition}{\textit{Fiber of $\Rn$.}}
For some $p\in\Rn$, the set $\left\{(p,v) : v\in\Rn\right\}$ formed by $\pi^{-1}(p)$ is defined as the \textbf{fiber} over $p$.
\end{definition}

\begin{remark}
The fiber $\pi^{-1}(p)$ can be pictured as all arrows that start at $p$. The name ``fiber'' is also intuitive: imagine a cylindrical hairbrush; $\pi (p,v)$ takes any point $v$ on some bristle on the brush and maps it to the root $p$ of that bristle. Such a fiber also forms a vector space, which will come up later.
\end{remark}

\begin{definition}{\textit{Directional Derivative}}
Let $f:\Rn\rightarrow\R^m$ be a differentiable map (for some $n, m$). The \textbf{directional derivative} $Df(p)(v)$ (the derivative of $f$ along $v\in\Rn$, at $p\in\Rn$) is defined by
\begin{equation}
    Df(p)(v)= \lim_{h\rightarrow 0} \frac{f(p+hv) - f(p)}{h}.
\end{equation}


\end{definition}

\begin{definition}{\textit{Pushforward Operator for $\Rn$.}}
Let $f:\Rn\rightarrow\R^m$ be a differentiable map.. The \textbf{pushforward operator} $f_*:T\Rn\rightarrow T\mathbb{R}^m$ is defined by
\begin{equation}
    f_*=\bigcup_{p\in\Rn}f_{*p},
\end{equation}
where $f_{*p}$ is the mapping that takes some $(p, v)\in T\Rn$ to $\left.Df(p)(v)\right|_{f(p)}\in T\R^m$.
\end{definition}

\begin{remark}
In other words, the pushforward operator $f_*$ takes a directional derivative operator (tangent vector) from within one tangent space $T\Rn$ to a directional derivative operator within another tangent space $T\R^m$, all dependent on the function $f:\Rn\rightarrow\R^m$. As such, $\pi\circ f_* = f \circ \pi$.
\end{remark}

\begin{definition}{\textit{Tangent Space on a Manifold.}}
    Let $M$ be some differentiable manifold with some chart $x:U\rightarrow\Rn$, and choose some $p\in M$. Suppose that we have two curves $c_1,c_2:(-\varepsilon,\varepsilon)\rightarrow M$ with $c_1(0)=c_2(0) = p$ for some $\varepsilon\in\R>0$, such that both $x\circ c_1, x\circ c_2:(-\varepsilon,\varepsilon)\rightarrow \Rn$ are differentiable.

    We say that $c_1$ and $c_2$ are \textbf{equivalent} at $0$ iff. these two derivatives coincide at $0$, defining an equivalence relation on the set of all differentiable curves initialized at $p$. This forms an equivalence class of such curves, known as the \textbf{tangent vectors} of $M$ at $p$. The \textbf{tangent space} of $M$ at $p$, denoted $T_pM$, is the set of all tangent vectors at $p$
\end{definition}

\begin{remark}
    Think of $T_p$ as the (vector) space of all possible directions through which $p$ can be passed through ``tangential'' to $M$. Note that $T_pM$ does \textit{not} depend on the choice of coordinate system $p$. 
\end{remark}

% \begin{remark}
%     The definition for such tangent spaces found within Spivak is too long for this sheet; as such
% \end{remark}


\begin{definition}{\textit{Vector Bundle.}}
An $n$-dimensional \textbf{vector bundle} is a five-tuple $\xi=(E,\pi,B,\oplus, \odot)$, where
\begin{enumerate}
    \item $E$ is the \textbf{total space}
    \item $B$ is the \textbf{base space}
    \item $\pi:E\rightarrow B$ is a continuous map onto $B$
    \item $\oplus$ and $\odot$ are operators defined to be analogous to addition and scalar multiplication, such that each \textit{fiber} $\pi^{-1}(p)$ ($p\in E$) is an $n$-dimensional vector space over $\R$.
\end{enumerate}
\end{definition}

\begin{remark}
Vector bundles, or bundles for short, are usually just notated $\pi: E\rightarrow B$, as the inclusion of $\oplus$ and $\odot$ to form the fiber vector spaces is treated as implicit.
\end{remark}

\begin{definition}{\textit{Tangent Bundle.}}
The \textbf{tangent bundle} $TM$ of a manifold $M$ is the (disjoint) union of all tangent spaces on $M$, i.e.
\begin{equation}
    TM = \bigcup_{p\in M}T_pM,
\end{equation}
and is a type of vector bundle, defined with $\pi: TM \rightarrow M$.
\end{definition}
 
\end{multicols*}
\end{document}