\documentclass[10pt,landscape]{article}
\usepackage{amssymb,amsmath,amsthm,amsfonts,commath,bm}
\usepackage{multicol,multirow}
\usepackage{physics}
\usepackage{tikz-cd}
\usepackage{hyperref}
\usepackage[landscape,margin=0.5in]{geometry}

\title{A Differential Geometry Basics Cheat Sheet}
\author{Nick Konz}

\setlength{\columnseprule}{0.4pt}


% styles
\theoremstyle{definition}
\newtheorem{definition}{Definition}[section]

\theoremstyle{theorem}
\newtheorem{theorem}{Theorem}[section]

\theoremstyle{summary}
\newtheorem*{summary}{Summary}

\theoremstyle{remark}
\newtheorem*{remark}{Remark}


% commands
\newcommand{\R}{\mathbb{R}}
\newcommand{\Rn}{\mathbb{R}^n}
\newcommand{\Ci}{C^\infty}

\begin{document}

\begin{center}
     \Large{\textbf{A Differential Geometry Basics Cheat Sheet}} \\
     \small{By Nick Konz}\\
     \textit{This is designed to be a quick, yet rigorous introduction/reference to the basic principles found within differential geometry, covering only the bare minimum of topics needed.}
     \small{\textit{Largely adapted from Spivak's} {A Comprehensive Introduction to Differential Geometry, 3rd Edition}, \textit{Mendelson's} Introduction to Topology, 3rd Edition and \textit{Michor's} Topics in Differential Geometry.}
\end{center}
\begin{multicols*}{3}

\section{Prerequisite Knowledge}
\textit{We only assume basic knowledge of continuity and differentiability in the context of functions and topology. Here, $\forall$ means ``for all'' or ``for each'', and $\exists$ means ``there exists''.}
\subsection{Some Basic Topology}

\theoremstyle{definition}
\begin{definition}{\textit{Neighborhood.}}
Let $a\in\Rn, \delta > 0$. The $\delta-$\textbf{neighborhood} of $a$ is the set
\begin{equation}
    U(a,\delta)=\left\{x\in\Rn:\norm{x-a}<\delta\right\}.
\end{equation}
\end{definition}

\theoremstyle{definition}
\begin{definition}{\textit{Metric Space.}}
Let $X$ be some non-empty set and $d$ be the mapping/function $d:X\times X\rightarrow \R$. The pair $(X,d)$ is a \textbf{metric space} if, $\forall x,y,z\in X$,
\begin{enumerate}
    \item $d(x,y) \geq 0$
    \item $d(x,y) = 0$ if and only if (iff) $x = y$
    \item $d(x,y)=d(y,x)$
    \item $d(x,z)\leq d(x,y) + d(y,z)$ \textit{(triangle inequality)}.
\end{enumerate}
\end{definition}

\theoremstyle{definition}
\begin{definition}{\textit{Topological Space.}}
Let $X$ be a non-empty set and $\mathcal{J}$ be a collection of subsets of $X$. $(X, \mathcal{J})$ is a \textbf{topological space} if
\begin{enumerate}
    \item $X \in \mathcal{J}$
    \item $\varnothing \in \mathcal{J}$, where $\varnothing$ is the empty set
    \item If $O_1, O_2, \ldots , O_n \in \mathcal{J}$, then $\bigcap_{i=1}^nO_i\in\mathcal{J}$
    \item If $\forall \alpha \in I$, $O_\alpha\in\mathcal{J}$, then $\bigcup_{\alpha\in I}O_\alpha\in\mathcal{J}$ \\($I$ is some \textit{indexing set}).
\end{enumerate}
We label $X$ as the \textbf{underlying set}, $\mathcal{J}$ as the \textbf{topology} on $X$, and members of $\mathcal{J}$ as \textbf{open sets}.

\end{definition}

\begin{remark}
Topological and metric spaces $(X,\mathcal{J})$ and $(X, d)$, respectively, are sometimes notated simply as $X$.
\end{remark}

\theoremstyle{definition}
\begin{definition}{\textit{Homeomorphism.}}
Let $(X,\mathcal{J})$ and $(Y, \mathcal{K})$ be topological spaces. $(X,\mathcal{J})$ and $(Y, \mathcal{K})$ are \textbf{homeomorphic} if $\exists$ inverse functions, or \textbf{homeomorphisms}, $f:X\rightarrow Y$ and $g:Y\rightarrow X$ such that $f, g$ are continuous.
\end{definition}

\subsection{Some Linear Algebra}
\theoremstyle{definition}
\begin{definition}{\textit{Vector Space.}}
A \textbf{vector space} $V$ is a set, or space, that is closed under addition and scalar multiplication, i.e. the result of performing these operations on some element(s) of $V$ is itself an element of $V$. We define elements of a vector space as \textbf{vectors}.
\end{definition}

\theoremstyle{definition}
\begin{definition}{\textit{Basis.}}
A \textbf{basis} for a vector space $V$ is a set of linearly independent vectors (i.e. none of the basis vectors can be written as linear combinations of the others) that span $V$ (i.e. any element of $V$ can be written as a linear combination of the basis vectors). In other words, the basis vectors define a ``coordinate system'' for $V$.
\end{definition}

\begin{remark}
You can convert vectors written in one basis to another via \textit{change-of-basis} matrices (or functions in the case of infinite-dimensional vectors).
\end{remark}

\theoremstyle{definition}
\begin{definition}{\textit{Dual Space.}}
Given some real vector space $V$, the \textbf{dual space} $V^*$ of $V$ is the vector space of all linear functions $f:V\rightarrow\R$.
\end{definition}

\begin{remark}
    Broadly, when some dual vector $f\in V^*$ acts on some vector $v\in V$, the scalar $fv=f(v)$ provides some information about/``measures'' something about $v$.
\end{remark}


\subsection{Some Discrete Math}
\begin{definition}{\textit{Equivalence Relation.}}
A \textbf{relation} on some set $X$ is a subset $R$ of $X\times X$. We write $x R y$ to mean $(x, y)\in R$, i.e. \textit{x is related to y}.
The relation is an \textbf{equivalence relation} if it is
\begin{enumerate}
    \item \textbf{reflexive}: $aRa \quad \forall a\in X$
    \item \textbf{symmetric}: $aRb \Rightarrow bRa \quad \forall a,b\in X$
    \item \textbf{transitive}: $aRb,\, bRc \Rightarrow aRc \quad \forall a,b,c\in X$.
\end{enumerate}
\end{definition}

\begin{definition}{\textit{Equivalence Class.}}
Given an equivalence relation $R$ on some set $X$, the \textbf{equivalence class} of some $y\in X$ is the set $\left\{x\in X: xRy\right\}$.
\end{definition}

\section{Manifolds}

\theoremstyle{definition}
\begin{definition}{\textit{Manifold.}}
A \textbf{manifold} is a metric space $M$ such that if $x\in M$, $\exists$ some neighborhood $U$ of $x$ and some $n\in\left\{0,1,2\ldots\right\}$ such that $U$ is homeomorphic to $\Rn$. If $\exists$ such an $n$ that is the same $\forall x\in M$, we say that $M$ is \textbf{$\bm{n}$-dimensional}, which can be notated as $M^n$.
\end{definition}

\begin{remark}
Think of a manifold as being a surface that is locally Euclidean.
\end{remark}

\theoremstyle{definition}
\begin{definition}{\textit{$C^\infty$-related Homeomorphisms.}}
Let $M$ be some manifold, and let $U$, $V$ be open subsets of $M$. Two homeomorphisms $x:U\rightarrow x(U)\subset\Rn$ and $y:V\rightarrow y(V) \subset \Rn$ (for some $n$) are $\bm{\Ci}$\textbf{-related} if the maps
\begin{align}
    y \circ x^{-1} &: x(U\cap V) \rightarrow y(U\cap V)\\
    x \circ y^{-1} &: y(U\cap V) \rightarrow x(U\cap V)
\end{align}
are infinitely differentiable, or $\bm{\Ci}$.
\end{definition}

\theoremstyle{definition}
\begin{definition}{\textit{Atlas.}}
A family of \textit{mutually} $\Ci$-related homeomorphisms whose domains cover $M$ (i.e. their union equals $M$) is an \textbf{atlas} of $M$.
\end{definition}

\theoremstyle{definition}
\begin{definition}{\textit{Chart/Coordinate System.}}
A \textbf{chart} or \textbf{coordinate system} for some manifold $M$ is a homeomorphism $x$ from some open $U\in M$ to an open subset of $\Rn$, denoted $(x, U)$. A chart of $M$ is a member of some atlas of $M$.
\end{definition}

\begin{remark}
Charts/coordinate systems $(x, U)$ create a way of assigning coordinates to points on $U$, and are sometimes notated simply with $x$.
\end{remark}

\theoremstyle{definition}
\begin{definition}{\textit{Differentiable Manifold.}}
A \textbf{differentiable}, \textbf{smooth} or $\bm{C^\infty}$ manifold is a pair $(M, \mathcal{A})$, where $M$ is some manifold, and $\mathcal{A}$ is some \textit{maximal atlas} for $M$, i.e. the union of all possible atlases of $M$. 
\end{definition}

\theoremstyle{definition}
\begin{definition}{\textit{Differentiable Map Between Manifolds.}}
Let $(M, \mathcal{A})$ and $(N,\mathcal{B})$ be differentiable/smooth manifolds (paired explicitly here with their atlases). A function $f:M\rightarrow N$ is \textbf{differentiable}/\textbf{smooth} if for all coordinate systems $(x,U)$ for $M$ and $(y,V)$ for $N$, the map
\begin{equation}
    y\circ f\circ x^{-1} : \Rn \rightarrow \R^m,
\end{equation}
where $n$ and $m$ are the dimensionalities of Euclidean space that $U$ and $V$ are homeomorphic to, respectively. We use $'$ to notate differentiation, i.e. the derivative of some $f$ is $f'$.
% add diffeomorphisms? only if needed later; this should be simple
\end{definition}

\begin{summary}
To summarize, a manifold is essentially a space that can be covered with coordinate charts, which are invertible, continuous mappings to some subset of $\Rn$. If the mapping between any pair of overlapping charts is differentiable, then the manifold itself is differentiable. 

We usually don't explicitly notate the atlas of a (differentiable) manifold, i.e. $(M, \mathcal{A})$ vs. simply $M$.
\end{summary}

\section{The Tangent Bundle}

\theoremstyle{definition}
\begin{definition}{\textit{Tangent Space on $\Rn$.}}
Consider some point $v\in\Rn$, drawn as an arrow with a ``reference point'' of some $p\in\Rn$; this arrow from $p$ to $p+v$ is denoted $(p,v)$. The set of all such $(p,v)$ is the \textbf{tangent space} $T\Rn = \Rn\times\Rn$ of $\Rn$. Elements of $T\Rn$ are \textbf{tangent vectors} of $\Rn$.
\end{definition}

\theoremstyle{definition}
\begin{definition}{\textit{Projection Map on $\Rn$.}}
The \textbf{projection map} $\pi:\Rn\times\Rn\rightarrow\Rn$ recovers the first member of any pair $(p, v)\in T\Rn$, and is defined as
\begin{equation}
    \pi(p, v) = p.
\end{equation}
\end{definition}

\begin{remark}
The projection map can be thought of as mapping any tangent vector to ``where it's at''/``where it comes from''.
\end{remark}

\theoremstyle{definition}
\begin{definition}{\textit{Fiber of $\Rn$.}}
For some $p\in\Rn$, the set $\left\{(p,v) : v\in\Rn\right\}$ formed by $\pi^{-1}(p)$ is defined as the \textbf{fiber} over $p$.
\end{definition}

\begin{remark}
The fiber $\pi^{-1}(p)$ can be pictured as all arrows that start at $p$. The name ``fiber'' is also intuitive: imagine a cylindrical hairbrush; $\pi (p,v)$ takes any point $v$ on some bristle on the brush and maps it to the root $p$ of that bristle. Such a fiber also forms a vector space, which will come up later.
\end{remark}

\theoremstyle{definition}
\begin{definition}{\textit{Directional Derivative.}}
Let $f:\Rn\rightarrow\R^m$ be a differentiable map (for some $n, m$). The \textbf{directional derivative} $Df(p)(v)$ (the derivative of $f$ along $v\in\Rn$, at $p\in\Rn$) is defined by
\begin{align}
    &Df(p)(v) = \lim_{h\rightarrow 0} \frac{f(p+hv) - f(p)}{h}. \\
    &= \left(D_1f(p)(v), \ldots, D_if(p)(v), \ldots ,D_nf(p)(v)\right)
\end{align}


\end{definition}

\theoremstyle{definition}
\begin{definition}{\textit{Pushforward Operator on $\Rn$.}}
Let $f:\Rn\rightarrow\R^m$ be a differentiable map. The \textbf{pushforward operator} $f_*:T\Rn\rightarrow T\mathbb{R}^m$ is defined by
\begin{equation}
    f_*=\bigcup_{p\in\Rn}f_{*p},
\end{equation}
where $f_{*p}$ is the mapping that takes some $(p, v)\in T\Rn$ to $\left.Df(p)(v)\right|_{f(p)}\in T\R^m$.
\end{definition}

\begin{remark}
In other words, the pushforward operator $f_*$ takes a directional derivative operator (tangent vector) from within one tangent space $T\Rn$ to a directional derivative operator within another tangent space $T\R^m$, all dependent on the function $f:\Rn\rightarrow\R^m$. As such, $\pi\circ f_* = f \circ \pi$.
\end{remark}

\theoremstyle{definition}
\begin{definition}{\textit{Tangent Space on a Manifold.}}
    Let $M$ be some differentiable manifold with some chart $x:U\rightarrow\Rn$, and choose some $p\in M$. Suppose that we have two curves $c_1,c_2:(-\varepsilon,\varepsilon)\rightarrow M$ with $c_1(0)=c_2(0) = p$ for some $\varepsilon\in\R>0$, such that both $x\circ c_1, x\circ c_2:(-\varepsilon,\varepsilon)\rightarrow \Rn$ are differentiable.

    We say that $c_1$ and $c_2$ are \textbf{equivalent} at $0$ iff. these two derivatives coincide at $0$, defining an equivalence relation on the set of all differentiable curves initialized at $p$. This forms an equivalence class of such curves, known as the \textbf{tangent vectors} of $M$ at $p$. The class of any such curve $c$ is written $c'(0)$. The \textbf{tangent space} of $M$ at $p$, denoted $T_pM$, is the set of all tangent vectors at $p$.
\end{definition}

\begin{remark}
    Think of $T_p$ as the (vector) space of all possible directions through which $p$ can be passed through ``tangential'' to $M$. Note that $T_pM$ does \textit{not} depend on the choice of coordinate system $x$. 
\end{remark}

\begin{remark}
Vector bundles, or bundles for short, are usually just notated $\pi: E\rightarrow B$, as the inclusion of $\oplus$ and $\odot$ to form the fiber vector spaces is treated as implicit.
\end{remark}

\theoremstyle{definition}
\begin{definition}{\textit{Pushforward Operator on Manifolds.}}
    \label{def_pushforward}
    Let $f:M\rightarrow N$ be a differentiable/smooth map between smooth manifolds $M$ and $N$, and let $p\in M$ (so then $f(p)\in N$). Then, the pushforward operator (at $p$) is given by
    \begin{equation}
        f_{*p}(c'(0)) = (f\circ c)'(0),
    \end{equation}
    where $c$ is a curve along $M$ with $c(0) = p$ (see the previous definition of tangent spaces).
\end{definition}

\begin{remark}
    If some $f:M\rightarrow N$ is a mapping between some manifolds $M$ and $N$, then the pushforward $f_*:M\rightarrow N$ is a mapping between their tangent spaces, at $p\in M$ and $f(p)\in N$, respectively.
\end{remark}

\theoremstyle{definition}
\begin{definition}{\textit{Vector Bundle.}}
    An $n$-dimensional \textbf{vector bundle} is a five-tuple $\xi=(E,\pi,B,\oplus, \odot)$, where
    \begin{enumerate}
        \item $E$ is the \textbf{total space}
        \item $B$ is the \textbf{base space}
        \item $\pi:E\rightarrow B$ is a continuous map onto $B$
        \item $\oplus$ and $\odot$ are operators defined to be analogous to addition and scalar multiplication, such that each \textbf{fiber} $\pi^{-1}(p)$ ($p\in E$) is an $n$-dimensional vector space over $\R$.
    \end{enumerate}
    \end{definition}
    
    \begin{remark}
    Vector bundles, or bundles for short, are usually just notated $\pi: E\rightarrow B$, as the inclusion of $\oplus$ and $\odot$ to form the fiber vector spaces is treated as implicit.
    \end{remark}


\theoremstyle{definition}
\begin{definition}{\textit{Bundle Map.}}
    Consider some bundles $\xi_1$ and $\xi_2$, defined with $\pi_1: E_1\rightarrow B_1$ and $\pi_2: E_2\rightarrow B_2$, respectively. A \textbf{bundle map} from $\xi_1$ to $\xi_2$ is a pair of continuous maps $(\overline{f}, f)$ with
    \begin{equation}
        \overline{f}:E_1\rightarrow E_2 \qquad \text{and} \qquad f:B_1\rightarrow B_2
    \end{equation}
    such that:
    \begin{enumerate}
        \item The following diagram commutes:
            \centering
            \begin{tikzcd}
                E_1 \arrow[r, "\overline{f}"] \arrow[d, "\pi_1"] & E_2 \arrow[d, "\pi_2"] \\
                B_1 \arrow[r, "f"] & B_2
            \end{tikzcd}
        \item $\overline{f}:\pi_1^{-1}(p)\rightarrow f:\pi_2^{-1}(p)$ is a linear map for some $p\in B_1$.
    \end{enumerate}
\end{definition}

\theoremstyle{definition}
\begin{definition}{\textit{Tangent Bundle.}}
    The \textbf{tangent bundle} $TM$ of a manifold $M$ is the (disjoint) union of all tangent spaces on $M$, i.e.
    \begin{equation}
        TM = \bigcup_{p\in M}T_pM,
    \end{equation}
    and is a type of vector bundle, defined with $\pi: TM \rightarrow M$.
\end{definition}

\theoremstyle{definition}
\begin{definition}{\textit{Derivation.}}
    For some manifold $M$, we can define a tangent vector at $p\in M$ to be a linear operator $\ell$, called a \textbf{derivation} at $p$, that operates on all $\Ci$ functions $f,g,\ldots$ such that
    \begin{equation}
        \ell(fg)(p) = f(p)\ell(g)(p) + g(p)\ell(f)(p).
    \end{equation}
\end{definition}

\theoremstyle{definition}
\begin{definition}{\textit{Partial Derivative.}}
   Let $f:M\rightarrow \R$ for some smooth manifold $M$ with coordinate chart $(x,U)$. The \textbf{partial derivative} operator $\ell = \displaystyle\left.\pdv{}{x^i}\right|_p$ is a derivation, defined as
   \begin{equation}
    \displaystyle\left.\pdv{}{x^i}\right|_p = {\pdv{f}{x^i}} (p) = D_i(f\circ x^{-1})(x(p)).
   \end{equation}
\end{definition}

\theoremstyle{theorem}
\begin{theorem}{\textit{Partial Derivatives as a Basis for the Tangent Space.}}
\label{thm_partialbases}
   Let $M^n$ be some $n-$dimensional smooth manifold, and let $(x,U)$ be some coordinate system around some $p\in M^n$. The set of all linear derivations at $p$ is an $n-$dimensional vector space, that is spanned by
   the partial derivatives
   \begin{equation}
    \displaystyle\left.\pdv{}{x^1}\right|_p,\ldots,\displaystyle\left.\pdv{}{x^n}\right|_p.
   \end{equation}
   This set is equivalent to $T_pM$, and the set of all derivations at all points of $M$ is equivalent to the tangent bundle $TM$. As such, \textbf{the partial derivatives form a basis for the tangent space}.
\end{theorem}

\theoremstyle{definition}
\begin{definition}{\textit{Section.}}
    Let $\xi$ be some vector bundle defined by $\pi:E\rightarrow B$. Then a \textbf{section} of $\xi$ is a continuous map $s:B\rightarrow E$ such that 
    \begin{equation}
        \pi(s(x)) = x \quad \forall x\in B.
    \end{equation}
\end{definition}

\theoremstyle{definition}
\begin{definition}{\textit{Vector Field on a Manifold.}}
\label{def_vectorfield}
    A \textbf{vector field} on a smooth manifold $M$ is a smooth map 
    \begin{equation}
        X:M\rightarrow TM
    \end{equation}
    such that for any $p\in M, X(p) \in T_pM$. \textbf{A vector field is a section of a tangent bundle}.
\end{definition}

\begin{remark}
    %A section $s$ of a bundle maps from the base space to the total space, picking out a point on the fiber over each point $p$ on the base. 
    In the context of manifolds, as sections pick out points in the total space that lie above the point on the manifold they're mapping from, we can project back down with $\pi$ to recover the original point. Going back to the hairbrush analogy, if the projection map maps any point along a brush fiber to its base, then the section maps the point on the base to some point on the fiber.
\end{remark}

\begin{summary}
    A tangent space is a vector space of all directional derivative operators at a given point on a manifold $M$. The tangent bundle is the (disjoint) union of all of the tangent spaces on the manifold, denoted $TM$. For some point on a manifold, the partial derivative operators (with respect to the coordinate system about that point) form a basis for the point's tangent space.
\end{summary}

\section{The Cotangent Bundle}

\begin{remark}
    All of the constructions that went into the tangent bundle $TM$ are built upon the fibers $\pi^{-1}(p)$, the vector space of all of the directional derivative operators passing through some $p\in M$. If we replace each $\pi^{-1}(p)$ with its dual space, then combine all of these vector spaces over $M$ as we did with the tangent spaces to make the tangent bundle, we will get another useful vector bundle.
\end{remark}

\theoremstyle{theorem}
\begin{theorem}{\textit{Bases of Dual Spaces.}}
\label{thm_dualbases}
    Consider some finite dimensional vector space $V$ with dual $V^*$. The dimension of $V^*$ equals that of $V$. Furthermore, if $v_1\ldots v_n$ is a basis for $V$, then the elements $v_i^*\in V^*$ defined by
    \begin{equation}
        v^*_i(v_j) = \delta_j^i = 
            \begin{cases}
                1 & i = j\\
                0 & i \neq j
            \end{cases}
    \end{equation}
    form a basis for $V^*$.
\end{theorem}

\theoremstyle{definition}
\begin{definition}{\textit{Dual Vector Bundle.}}
    \label{def_dualbundle}
    Let $\xi$ be some vector bundle defined by $\pi: E\rightarrow B$, and let
    \begin{equation}
        E'=\bigcup\limits_{p\in B}\left[\pi^{-1}(p)\right]^*.
    \end{equation}
    If we define $\pi':E'\rightarrow B$ to take each $\left[\pi^{-1}(p)\right]^*$ to $p$, then we can construct a \textbf{dual bundle} $\xi^*$ from $\pi'$, whose fibers are $\left[\pi^{-1}(p)\right]^*$, the dual spaces to the fibers of $\xi$.
\end{definition}

\theoremstyle{definition}
\begin{definition}{\textit{Cotangent Bundle.}}
    \label{def_cotangentbundle}
    Let $M$ be some smooth manifold with tangent bundle $TM$. Then the \textbf{cotangent bundle} of $M$, $T^*M$, is the dual bundle of $TM$. A fiber of $T^*M$ over some $p\in M$ is $T_p^*M = \left[\pi^{-1}(p)\right]^*$, known as a \textbf{cotangent space} at $p$.
\end{definition}

\theoremstyle{definition}
\begin{definition}{\textit{Differential.}}
\label{def_differential}
    Consider some smooth manifold $M$ and some smooth function $f:M\rightarrow \R$, with a vector field $X(p)\in T_pM$ given some $p\in M$. A $\Ci$/smooth section $df$ of $T^*M$ called the \textbf{differential} of $f$ can be defined by
    \begin{equation}
        df(p)(X) = X(f).
    \end{equation}
\end{definition}

\theoremstyle{theorem}
\begin{theorem}{\textit{Differentials as Bases for Cotangent Spaces.}}
    \label{thm_differentialsbasiscotangentspaces}
    Consider some smooth manifold $M$ with coordinate chart $(x, U)$ for some $p\in M$, with $x = (x^1,\ldots,x^n)$. The differentials $dx^i$ are sections of $T^*M$ over $U$.
    From Theorems \ref{thm_partialbases} and \ref{thm_dualbases}, then 
    \begin{equation}
        dx^i(p)\left(\displaystyle\left.\pdv{}{x^j}\right|_p\right) = \delta_j^i,
    \end{equation}
    i.e. the differentials
    \begin{equation}
        dx^1(p), \ldots, dx^n(p)
    \end{equation}
    form a basis of $T^*_pM$, dual to the basis $\left.\pdv{}{x^1}\right|_p,\ldots,\left.\pdv{}{x^n}\right|_p$ of $T_pM$. As such, given some smooth function $f:M\rightarrow \R$, the section $df$ of $T^*_pM$ can be expressed as
    \begin{equation}
        df = \sum\limits_{i=1}^n\pdv{f}{x^i}dx^i,
    \end{equation}
    given Definitions \ref{def_vectorfield} and \ref{def_differential}.
\end{theorem}

\theoremstyle{definition}
\begin{definition}{\textit{Pullback Operator (on a Manifold).}}
    Let $f:M\rightarrow N$ be a differentiable/smooth map between smooth manifolds $M$ and $N$, and let $p\in M$. Recall from Definition \ref{def_pushforward} that $\exists$ a pushforward operator $f_*:TM\rightarrow TN$ defined for a single $p\in M$ as $f_{*p} : T_pM\rightarrow T_{f(p)}N$. As $f_{*p}$ is a linear transformation between vector spaces, taking the dual of the spaces gives rise to the map between cotangent spaces
    \begin{equation}
        f_p^* : T_{f(p)}^* N\rightarrow T_p^*M,
    \end{equation}
    which we label as the \textbf{pullback operator}.
\end{definition}

\begin{remark}
    Note that we can't just pull all of the $f_p^*$ together over all $p\in M$ as we did for the pushforward $f_{*p}$ to obtain a mapping between the cotangent bundles $T^*N$ and $T^*M$ (see Spivak pg. 113).
\end{remark}

\theoremstyle{theorem}
\begin{theorem}{\textit{Section of a Cotangent Space.}}
    Let $f:M\rightarrow N$ be a differentiable/smooth map between smooth manifolds $M$ and $N$. Given some section $\omega$ of $T^*N$, we can define a section $\eta$ of $T^*M$ as
    \begin{equation}
        \eta(p) = \omega (f(p)) \circ f_{*p}
    \end{equation}
    for some $p\in M$, i.e. for some vector field/section of $TM$, $X$,
    \begin{equation}
        \eta(p)X(p) = \omega(f(p))(f_{*p}X(p)).
    \end{equation}
\end{theorem}

\begin{remark}
    Intuitively what this means is that to operate on some vector at $p$ in $TM$, we push it over to $TN$ via $f_*$, then operate on it with $\omega$. 
    
    In summary, the \textit{pushforward} $f_{*p}$ maps from the tangent space of $M$ at $p$ to the tangent space of $N$ at $f(p)$. In turn, the \textit{pullback} $f_p^*$ maps from the space of sections of the cotangent bundle of $N$ to the space of sections of the cotangent bundle onf$M$.
\end{remark}

\theoremstyle{definition}
\begin{definition}{\textit{Contravariant and Covariant Vector Fields.}}
    A \textbf{contravariant vector field} is a vector field on some smooth manifold $M$, i.e. a section of $TM$. A \textbf{covariant vector field} is a section of $T^*M$.
\end{definition}

\begin{remark}
    Covariant and contravariant vector fields, i.e. sections of $T^*M$ and $TM$, respectively, are also known as covariant and contravariant \textit{tensors} (or \textit{tensor fields}) of order $1$. We will dive into tensors in the following section.
\end{remark}

\begin{summary}
    For some smooth manifold $M$, the cotangent bundle $T^*M$ is made out of the duals of the fibers of the tangent bundle $TM$, for all $p\in M$. Differentials are sections of $T^*M$, and form bases for cotangent spaces at points on $M$, just as their duals, partial derivatives, form bases for the corresponding tangent spaces. Just as the pushforward operator is used to map between tangent spaces of different manifolds, the pullback operator is used to map between spaces of sections of cotangent bundles. Contravariant vector fields are sections of $TM$, while covariant vector fields are sections of $T^*M$.
\end{summary}

\section{Tensors}

\theoremstyle{theorem}
\begin{theorem}{\textit{Coordinate Transformations of Differentials.}}
    Consider some linear coordinate system $x$ on $\Rn$ (i.e. $x$ is defined solely by linear transformations). If $x'$ is another such linear coordinate system, then by the definition of linear transformations, $x'^j=\sum\limits_{i=1}^na_{ij}x^i$ for some $a_{ij}$. In fact, $a_{ij} = \pdv{x'^j}{x^i}$ (as $\pdv{x'^j}{x^i}$ are elements of the Jacobian matrix $D(x'\circ x^{-1})$), so that
    \begin{equation}
        x'^j=\sum\limits_{i=1}^n\pdv{x'^j}{x^i}x^i.
    \end{equation}
    Then, from Theorem \ref{thm_differentialsbasiscotangentspaces}, we have that
    \begin{equation}
        dx'^j=\sum\limits_{i=1}^n\pdv{x'^j}{x^i}dx^i.
    \end{equation}
\end{theorem}

\begin{remark}
    Essentially what the preceding theorem means is that the differentials $dx^i$ change in the ``same way'' as the coordinates $x^i$, described as being \textbf{covariant}.
\end{remark}

\theoremstyle{definition}
\begin{definition}{\textit{Multilinear Function.}}
    Let $V_1,\ldots ,V_m$ be some collection of vector spaces. A function
    \begin{equation}
        T:V_1\times\cdots\times V_M\rightarrow \R
    \end{equation}
    is \textbf{multilinear} if $\forall i$ of $V_i$, if all input variables but $v_i\in V_i$ are held constant, $T$, i.e. $T(v_1,\ldots v_i,\ldots,v_m)$, is a linear function of $v_i$. In other words, the mapping
    \begin{equation}
        v_i\rightarrow T(v_1,\ldots,v_{i-1}, v_i, v_{i+1},\ldots,v_m)
    \end{equation}
    is linear for each choice of $v_1,\ldots,v_{i-1}, v_{i+1},\ldots,v_m$. The set of all such $T$ is a vector space.

    If $V_i=V$ $\forall i$ for some $V$, this vector space is denoted $\mathcal{T}^m(V)$, the \textbf{space of all multilinear functions on $\bm{V}$}.
\end{definition}

\begin{remark}
    Note that $\mathcal{T}^1(V) = V^*$.
\end{remark}

\theoremstyle{definition}
\begin{definition}{\textit{Multilinear Pullback Operator.}}
    Let $V,W$ be vector spaces, and let $f:V\rightarrow W$ be some linear transformation. There exists a linear transformation $f^*\mathcal{T}^m(W)\rightarrow\mathcal{T}^m(V)$ defined as
    \begin{equation}
        f^*T(v_1,\ldots,v_m) = T(f(v_1),\ldots,f(v_m)),
    \end{equation}
    called the \textbf{multilinear pullback operator}.
\end{definition}

\theoremstyle{definition}
\begin{definition}{\textit{Tensor Product.}}
    Given some vector space $V$, for some $T\in\mathcal{T}^k(V)$ and $S\in\mathcal{T}^\ell(V)$, the \textbf{tensor product} $T\otimes S\in\mathcal{T}^{k+\ell}(V)$ is defined by
    \begin{align}
        & T\otimes S (v_1,\ldots,v_k,v_{k+1},\ldots,v_{k+\ell}) \\
        & = T(v_1,\ldots,v_k)\cdot S(v_{k+1},\ldots,v_{k+\ell}).
    \end{align}
\end{definition}

\theoremstyle{theorem}
\begin{theorem}{\textit{Bases for Multilinear Function Spaces.}}
    \label{thm_basismultilinear}
    Consider a vector space $V$ with basis $v_1,\ldots,v_n$, and let $T, S, U$ be some multilinear functions defined on $V$. Note that the tensor product $\otimes$ is associative, i.e. $(S\otimes T)\otimes U = S\otimes (T\otimes U)$. 
    
    If $v_1^*,\ldots v_n^*$ is the dual basis for $V^*=\mathcal{T}^1(V)$, then the elements
    \begin{equation}
        v_{i_1}^*\otimes\cdots\otimes v_{i_k}^* \qquad 1\leq i_1,\ldots,i_k\leq n,
    \end{equation}
    with all possible $\binom{n}{k}$ choices for $(i_1,\ldots,i_k)$ from $(1,\ldots, n)$, form a basis for $\mathcal{T}^k(V)$, which has dimension $n^k$, using the aforementioned associativity to define such an $k$-fold tensor product.
\end{theorem}

\begin{remark}
    Similarly to how we used the dual space to build the cotangent bundle (Definitions \ref{def_dualbundle} and \ref{def_cotangentbundle}), we can use this multilinear function space to define a new bundle.
\end{remark}

\theoremstyle{definition}
\begin{definition}{\textit{Covariant Tensor Bundle.}}
    Let $\xi$ be some vector bundle defined by $\pi: E\rightarrow B$, and let
    \begin{equation}
        E'=\bigcup\limits_{p\in B}\mathcal{T}^k\left(\pi^{-1}(p)\right).
    \end{equation}
    If we define $\pi':E'\rightarrow B$ to take each $\mathcal{T}^k\left(\pi^{-1}(p)\right)$ to $p$, then we can construct a bundle $\mathcal{T}^k(\xi)$ defined by $\pi':E'\rightarrow B$.

    Considering the tangent bundle $TM$ of some smooth manifold $M$, the bundle $\mathcal{T}^k(TM)$ is called the \textbf{covariant tensor bundle of order $\bm{k}$}.
\end{definition}

\begin{remark}
    From the preceding definition, note that the dual bundle $\xi^*$ is the special case of $k=1$ for $\mathcal{T}^k(\xi)$.
\end{remark}

\theoremstyle{theorem}
\begin{theorem}{\textit{Products of Differentials as Bases for $\mathcal{T}^k(T_pM)$.}}
    \label{thm_basesforcovariantspacepoint}
    Given some smooth manifold $M$, consider some coordinate system $(x ,U)$ about some $p\in M$. From Theorem \ref{thm_differentialsbasiscotangentspaces}, then the differentials
    \begin{equation}
        dx^1(p), \ldots, dx^n(p)
    \end{equation}
    form a basis for the cotangent space $T_p^*M$. Then from Theorem \ref{thm_basismultilinear}, the $k$-fold tensor products
    \begin{align}
        & dx^{i_1}(p)\otimes\cdots\otimes dx^{i_k}(p)\in\mathcal{T}^k(T_pM),\\
        & 1\leq i_1,\ldots,i_k\leq n
    \end{align}
    form a basis for $\mathcal{T}^k(T_pM)$.
\end{theorem}

\theoremstyle{definition}
\begin{definition}{\textit{Covariant Tensor Field.}}
    Consider some smooth manifold $M$ with covariant tensor bundle $\mathcal{T}^k(TM)$. A section of $\mathcal{T}^k(TM)$ is called a \textbf{covariant tensor field of order $\bm{k}$}. From Theorem \ref{thm_basesforcovariantspacepoint}, given some coordinate system $(x ,U)$ about some $p\in M$ we can write every covariant tensor field $A$ of order $k$ as
    \begin{equation}
        A(p) = \sum\limits_{i_1,\ldots,i_k}A_{i_1,\ldots,i_k}(p)dx^{i_1}(p)\otimes\cdots\otimes dx^{i_k}(p)
    \end{equation}
    given some functions $A_{i_1,\ldots,i_k}$, or simply
    \begin{equation}
        A = \sum\limits_{i_1,\ldots,i_k}A_{i_1,\ldots,i_k}dx^{i_1}\otimes\cdots\otimes dx^{i_k},
    \end{equation}
    removing any explicit notation of coordinate system.
\end{definition}
 
\end{multicols*}
\end{document}